\documentclass[a4paper]{article}
% Preamble
\usepackage[utf8]{inputenc}
\usepackage{fullpage}
\usepackage[english]{babel}
\usepackage{color}
\usepackage{amsmath}
\usepackage{url}
\usepackage{standalone}
\usepackage{parskip}
\usepackage{graphicx}
\usepackage{caption}
\usepackage{subcaption}
\usepackage{natbib}
\usepackage{amsfonts}

\title{Geometric algebra}
\author{Lauren Shriver}
\date{March 25 2019}

\begin{document}

\maketitle

\section*{Geometric product $\vec{a}\vec{b}$}
    The \textbf{geometric product} is subject to the following identities:
    \begin{itemize}
        \item \textbf{Associative}: $(\vec{a}\vec{b})\vec{c}=\vec{a}(\vec{b}\vec{c})$
        \item \textbf{Left distributive}: $\vec{a}(\vec{b}+\vec{c})=\vec{a}\vec{b}+\vec{a}\vec{c}$
        \item \textbf{Right distributive}: $(\vec{b}+\vec{c})\vec{a}=\vec{b}\vec{a}+\vec{c}\vec{a}$
        \item \textbf{Contraction}: $\vec{a}^2=|\vec{a}|^2$
    \end{itemize}
    The geometric product gives rise to two new vector products:
    \begin{enumerate}
        \item \textbf{Inner product}: $\vec{a}\cdot \vec{b}$
            \begin{itemize}
                \item Symmetric
                \item Outputs a scalar
                \item Expression: $\vec{a}\cdot \vec{b} = \frac{1}{2}(\vec{a}\vec{b} + \vec{b}\vec{a}) = \vec{b}\cdot \vec{a}$
            \end{itemize}
        \item \textbf{Outer product}: $\vec{a} \wedge \vec{b}$
            \begin{itemize}
                \item Antisymmetric
                \item Outputs a bivector
                \item Expression: $\vec{a}\wedge \vec{b} = \frac{1}{2}(\vec{a}\vec{b}-\vec{b}\vec{a}) = -\vec{b}\wedge \vec{a}$
            \end{itemize}
    \end{enumerate}
    By \textbf{canonical decomposition}, these two product definitions allows us to write the following expression as a definition for the geometric product:
    \begin{equation*}
        \vec{a}\vec{b} = \vec{a}\cdot\vec{b} + \vec{a}\wedge\vec{b} = \text{inner product} + \text{outer product}
    \end{equation*}
\subsection*{Properties}
    \begin{itemize}
        \item For orthogonal vectors, $\vec{a}\cdot\vec{b}=0 \Leftrightarrow \vec{a}\vec{b}=-\vec{b}\vec{a}$
        \item For colinear vectors, $\vec{a}\wedge\vec{b}=0 \Leftrightarrow \vec{a}\vec{b}=\vec{b}\vec{a}$
            \begin{itemize}
                \item The geometric product of two colinear vectors corresponds with a parallelogram with zero area
            \end{itemize}
    \end{itemize}
\subsection*{Geometric interpretation}
    \begin{itemize}
        \item $\vec{a}\vec{b}$ quantifies/measures the relative direction of vectors $\vec{a}$ and $\vec{b}$
        \item A pair of colinear vectors are said to be \textbf{commutative}
        \item A pair of orthogonal vectors are said to be \textbf{anticommutive}
        \item Multiplication can be extended to these extreme cases by introducing an orthonormal set of vectors 
    \end{itemize}

\section*{Basis and Bivectors}
\subsection*{Review: The Kronecker delta}
The \textbf{Kronecker delta} is a \textbf{piecewise function} of variables $i$ and $j$
    \begin{equation*}
        \delta_{ij}=
            \begin{cases}
                1, &         \text{if } i=j,\\
                0, &         \text{if } i\neq j.
            \end{cases}
    \end{equation*}

\section*{Vectors and Complex Numbers}
\section*{Geometric Algebra for Physical Space}
\section*{Reflections and Rotations}
\end{document}
