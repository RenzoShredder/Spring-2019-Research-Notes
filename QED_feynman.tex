\documentclass[a4paper]{article}
% Preamble
\usepackage[utf8]{inputenc}
\usepackage{fullpage}
\usepackage[english]{babel}
\usepackage{color}
\usepackage{amsmath}
\usepackage{url}
\usepackage{standalone}
\usepackage{parskip}
\usepackage{graphicx}
\usepackage{caption}
\usepackage{subcaption}
\usepackage{natbib}
\usepackage{amsfonts}
    
\title{Notes on \textit{QED: The Strange Theory of Light and Matter}}
\author{Lauren Shriver}
\date{2/11/2019}

\begin{document}
	\maketitle
	\section*{Chapter 1 - Introduction}
	    \begin{itemize}
	        \item Sometimes, things that seem different are really all just different aspects of the same thing
	        \item Around the time Maxwell "synthesized phenomena of light and optics into a single theory" (1873), physics could be described in terms of...
	       \begin{itemize}
	           \item The laws of motion
	           \item The laws of electricity and magnetism
	           \item The laws of gravity
	       \end{itemize}
	       \item There are no good theories explaining \textit{why} nature behaves the way it does
	       \item Like other wave phenomena, light can be described quantitatively by its frequency 
	       \item Note: Feynman uses "red light" in his examples for describing QED. However, his examples apply to EM waves of all different frequencies 
	       \item \textbf{Photomultiplier} = a device used to detect single photons (makes a "click" noise every time a photon is detected)
	       \begin{itemize}
	           \item Dimming the light being detected by the photomulitplier will result in fewer "clicks" than bright light, but each click is equally as loud regardless of whether the light is bright or dim
	           \item "Thus, light is something like raindrops - each little lump of light is called a photon - and if the light is all one color, all the "raindrops" are the same size 
	       \end{itemize}
	    \end{itemize}
    \section*{Chapter 2 - Photons: Particles of Light}
        \begin{itemize}
            \item "Rules" for determining the probability a particular event will occur
            \begin{itemize}
                \item \textit{Grand Principle}: The \textbf{probability amplitude} is equivalent the the norm squared (i.e., the square of the length of a vector "arrow") of a particular vector and represents the probability some corresponding event that will occur
                \begin{itemize}
                    \item e.g., an vector arrow with a magnitude of 0.4 corresponds with an event that has a 16\% probability of occurring
                \end{itemize}
                \item \textit{General Rule}: In the context of "drawing vector arrows", follow the format...
                    \begin{itemize}
                        \item For an event that may occur in multiple ways,
                        \begin{itemize}
                            \item Draw an arrow for each way the event may occur
                            \item "Add" the vectors by arranging them in a "tip-to-tail" manner
                            \item Draw a final "summation arrow" originating from the origin in of the first vector in your vector-summation arrangement and terminating at the end point of the final vector 
                                \begin{itemize}
                                    \item The square of this final arrow gives the probability that the overall event will occur
                                \end{itemize}
                        \end{itemize}
                    \end{itemize}
            \end{itemize}
        \end{itemize}
\end{document}
	